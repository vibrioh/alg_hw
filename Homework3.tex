\documentclass[11pt]{article}
\usepackage[letterpaper,margin=1in]{geometry}
\usepackage{amsmath,amsbsy,amsfonts,amssymb,amsthm,commath}
\usepackage[round]{natbib}
\usepackage{hyperref}
\usepackage{graphicx}
\usepackage{float}

% math font macros

\def\ddefloop#1{\ifx\ddefloop#1\else\ddef{#1}\expandafter\ddefloop\fi}
% Blackboard fonts: \bbA, \bbB, ...
\def\ddef#1{\expandafter\def\csname bb#1\endcsname{\ensuremath{\mathbb{#1}}}}
\ddefloop ABCDEFGHIJKLMNOPQRSTUVWXYZ\ddefloop
% Calligraphic fonts: \cA, \cB, ...
\def\ddef#1{\expandafter\def\csname c#1\endcsname{\ensuremath{\mathcal{#1}}}}
\ddefloop ABCDEFGHIJKLMNOPQRSTUVWXYZ\ddefloop
% Bold fonts (for vectors, matrices, etc.): \vA, \vB, ..., \va, \vb, ...
\def\ddef#1{\expandafter\def\csname v#1\endcsname{\ensuremath{\boldsymbol{#1}}}}
\ddefloop ABCDEFGHIJKLMNOPQRSTUVWXYZabcdefghijklmnopqrstuvwxyz\ddefloop
% Bold fonts (for vectors, matrices, etc.): \valpha, \vbeta, ...,  \vGamma, \vDelta, ...,
\def\ddef#1{\expandafter\def\csname v#1\endcsname{\ensuremath{\boldsymbol{\csname #1\endcsname}}}}
\ddefloop {alpha}{beta}{gamma}{delta}{epsilon}{varepsilon}{zeta}{eta}{theta}{vartheta}{iota}{kappa}{lambda}{mu}{nu}{xi}{pi}{varpi}{rho}{varrho}{sigma}{varsigma}{tau}{upsilon}{phi}{varphi}{chi}{psi}{omega}{Gamma}{Delta}{Theta}{Lambda}{Xi}{Pi}{Sigma}{varSigma}{Upsilon}{Phi}{Psi}{Omega}{ell}\ddefloop

% other macros

\newcommand\ip[1]{\langle #1 \rangle} % inner product
\newcommand{\E}{\ensuremath{\mathbb{E}}} % expectation
\renewcommand{\P}{\ensuremath{\mathbb{P}}} % probability
\newcommand{\var}{\ensuremath{\operatorname{var}}} % variance
\newcommand{\vol}{\ensuremath{\operatorname{vol}}} % volume
\newcommand{\unitball}[1][d]{\ensuremath{B^{#1}}} % unit ball
\newcommand{\unitsphere}[1][d-1]{\ensuremath{S^{#1}}} % unit sphere
\newcommand{\logmgf}[1]{\ensuremath{\psi_{#1}}} % log mgf
\newcommand{\Normal}{\ensuremath{\operatorname{N}}} % normal distribution

% environments

\newtheorem{theorem}{Theorem}
\newtheorem{lemma}{Lemma}
\theoremstyle{definition}
\newtheorem{problem}{Problem}
\newenvironment{solution}{\noindent\emph{Solution.}}{\hfill$\square$}

%-------------------------------------------------------------------------------
% define my own command:
\newcommand\tab[1][1cm]{\hspace*{#1}}

\usepackage{listings} %For code in appendix
  \lstset{language=Java, numbers=left, showspaces=false, %Set code style
    showstringspaces=false, tabsize=2, breaklines=true}



\begin{document}

%-------------------------------------------------------------------------------
\begin{center}
\Large{} 
\textbf{CSORW4231 HOMEWORK 3} \\
\normalsize{}
Due Mon, Mar 06 \\
\large{Jun Hu \\
\textbf{(jh3846)}} \\ 
------------------------------------------------------------------------------------------------------------------
\end{center}
%-------------------------------------------------------------------------------

\begin{problem}
\large{Exercise 8.1-1 on page 193 and Exercise 8.2-4 on page 197. }
\end{problem}

\begin{solution}
\begin{enumerate}
    \item[\textbf{8.1-1}]
    If the array is already sorted, we only need to compare $n-1$ times between all $n$ elements. So the smallest possible depth for the decision tree is $n-1$. Draw decision tree:
    \begin{figure}[htbp]
  \centering
  \includegraphics[width=0.5\textwidth]{/Users/vibrioh/Pictures/hw3_1}
  \caption{Decision Tree of 8.1-1}
  \label{fig:shapes}
\end{figure}
    
    
    
    
    \item[\textbf{8.2-4}]
    Borrow the method from COUNTING-SORT by creating a count set $C$ for query.
    Assuming given n integer set is $A$.
    \begin{lstlisting}
COUNTING(A)
		Let C[0..k] be a new array
		for i == 0 to k
			C[i] = 0
		for j == 1 to n
			C[A[j]] = C[A[j]] + 1
		for i = i to k
			C[i] = C[i] + C[i + 1]	
\end{lstlisting}
The processing for COUNTING is:
$$T_1 = c_1k + c_2n + c_3k + c_4 = \Theta(n + k)$$

When given query input [a, b]:
 \begin{lstlisting}
QUERY(a, b)
		return C[min(b, k)] - C[max(a - 1, 0)]
\end{lstlisting}
The query returning time is:
$$T_2 = c_5 = \Theta(1)$$





\end{enumerate}

\end{solution}

\newpage

%-------------------------------------------------------------------------------

%-------------------------------------------------------------------------------

\begin{problem}
Problem 8-4(b) on page 207 and 8-6(a and b) on page 208: Comparison lower bound. (If you would like to challenge yourself, check Exercise 9.1-2 on page 215.) 
\end{problem}

\begin{solution}
\begin{enumerate}
    \item[\textbf{8-4 b.}]
    Taking one red jar to compare with blue jars to find a matched pair can be looked as a decision tree with $3$ children at each node.\\
    The lower bound of Lower bound or the number of comparisons is the lower bound on the height of the decision tree.\\
    Observation:
    $$\mbox{Number}_{leaves} \geq n!$$
    Let h be the height of the decision tree:
    $$3^h \geq \mbox{Number}_{leaves} $$
    Which is:
    $$3^h \geq n!$$
    Stirling Approximation:
    $$n! = \sqrt[]{2\pi n}\left(\frac{n}{e}\right)^n\left(1 + \Theta\left(\frac{1}{n}\right)\right)$$
    Thus:
    \begin{align*}
    h &\geq \log_3(n!) \\
    &\geq \log_3\left(\sqrt[]{2\pi n}\left(\frac{n}{e}\right)^n\left(1 + \Theta\left(\frac{1}{n}\right)\right)\right) \\
   &\geq \log_3\left(\frac{n}{e}\right)^n \\
    &= n\log_3n - n\log_3e\\
    &=\Omega(n\lg n)
    \end{align*}
    
    
    

    \item[\textbf{8-6 a.}]
    The possible ways are picking $n$ from $2n$ for combination:
    $$\binom nk$$
    
    \item[\textbf{8-6 b.}]
    $$2^h \geq \mbox{Number}_{leaves} \geq \binom nk$$
    \begin{align*}
    h &\geq \lg \binom nk \\
    &\geq \lg\left(  \frac{2^{2n}}{\sqrt[]{\pi n}} \left(       1 + O\left(    \frac{1}{n}    \right)         \right)      \right) \\
    &= \lg(2^{2n}) - \lg \sqrt[]{\pi n} + \lg \left(       1 + O\left(    \frac{1}{n}    \right)    \right)   \\
    &= 2n - o(n)
    \end{align*}

  \end{enumerate}
\end{solution}

\clearpage


%-------------------------------------------------------------------------------


%-------------------------------------------------------------------------------

%-------------------------------------------------------------------------------

\begin{problem}
Problem 8-2 on page 206: Sorting in place in linear time. 4. Exercise 9.3-6 and Exercise 9.3-8 on page 223.
\end{problem}

\begin{solution}
\begin{enumerate}
    \item[\textbf{8-2 a.}]
    \
        \begin{lstlisting}
SORT(A)
		Let C[1..n] be a new array
		j == 1
		for i = 1 to n
			if A[i] == 0
				C[j] == A[i]
			j = j + 1
		for j = 1 to n
			if A[i] == 1
				C[j] == A[i]
			j = j + 1
		return C
\end{lstlisting}
The elements in A are collected into C keeping the same odder in A with the same key value 0 or 1, which mean the sort is stable.
The array A has been visited twice in the algorithm, so $T = O(n)$

    \item[\textbf{8-2 b.}]
    \
        \begin{lstlisting}
SORT(A)
		j == 1
		for i = 1 to n
			if A[i] == 0
				exchange A[i] with A[j]
				j = j + 1
		return A
\end{lstlisting}
Only maintaining A and taking $O(n)$.

    


    \item[\textbf{8-2 c.}]
    \
            \begin{lstlisting}
SORT(A)
	for i = 1 to n - 1
		for j = n down to i + 1
			if A[j] < A[j - 1]
				exchange A[j] with A[j - 1]
	return A
\end{lstlisting}



\item[\textbf{8-2 d.}]
Use algorithm in (a), which takes O(n) time.
A b-bit key takes bit value as 0 or 1, we can sort in the running time of $$ O(b(n + 2)) = O(bn)$$ 

\item[\textbf{8-2 e.}]
\
\begin{lstlisting}
SORT(A)
	Let C[1..k] be a new array
	for i = 1 to k
		C[i] == 0
	for i = 1 to n
		C[A[i]] = C[A[i]] + 1
	for i = 2 to k
		C[i] = C[i] + C[i - 1
	j == 1
	while j < n
		A.copy = A[j]
		if j != C[A.copy]
			exchange A[C[A.copy]] with A[j]
			C[A.copy] = C[A.copy] -1
		else j = j + 1
	return A
\end{lstlisting}
 The modified COUNTING-SORT is not stable.

\item[\textbf{9.3-6}]
QT(A, k) \\
Let C be a new array with length k - 1 \\
if k == 1 \\
	\tab return \\
else \\
	\tab i = $\lfloor \frac{k}{2} \rfloor$ \\
	\tab q = SELECT(A,$\lfloor i \frac{n}{k} \rfloor $ ) \\
	\tab PARTITION(A, q) \\
	\tab C.append(QT(A[1] to A[$\lfloor i \frac{n}{k} \rfloor $], $\lfloor \frac{k}{2} \rfloor$, C )) \\
	\tab C.append(QT(A[$\lfloor i \frac{n}{k} \rfloor $ + 1 ] to A[n], $\lceil \frac{k}{2} \rceil$, C )) \\
	\tab return q \\
Draw a recursion tree. At the top level we need to find k − 1
order statistics, and it costs O(n) to find one. For each of the two roof contains at the most $\lfloor \frac{k - 1}{2} \rfloor$ and $\lceil \frac{k - 1}{2} \rceil$order statistics The sum of the costs for these two nodes is O(n). The level of the tree is $\lg (k - 1)$, n numbers for each level, so the running time is $\Theta(n\lg k)$.

\item[\textbf{9.3-8}]
\
\begin{lstlisting}
MEDIAN(X, Y, n)
if n == 1
	return min(X[1], Y[1])
if X[n/2] < Y[n/2]
	return MEDIAN(X[n/2 + 1 ... n], Y[1 ... n/2], n/2)
else
	return MEDIAN(X[1 ... n/2], Y[n/2 + 1 .. n], n/2)	
\end{lstlisting}


\end{enumerate}

\end{solution}

\newpage

%-------------------------------------------------------------------------------

%-------------------------------------------------------------------------------

\begin{problem}Problem 9.2(skip a and b) on page 225: Weighted median. 
\end{problem}

\begin{solution}
  \begin{enumerate}
  
    \item[\textbf{9.2 c.}]
Modify SELECT as:
Let m be the median of median,
Compute $\sum_{m_i < m} w_i$ and $ \sum_{m_i > m} w_i $,
Check either exceeds 1/2,
If so, recursive call for the one contain weighted median
If not, return.
Sp the running time keeps 
$\Theta(n)$

   \item[\textbf{9.2 d.}]
   
   
   \item[\textbf{9.2 e}]


    
    
  \end{enumerate}
\end{solution}

\newpage
%-------------------------------------------------------------------------------

%-------------------------------------------------------------------------------

%-------------------------------------------------------------------------------

\begin{problem}
Problem 11-4(a and b only) on page 284. 
\end{problem}

\begin{solution}
\begin{enumerate}
   \item[\textbf{11-4 a.}]
   
      \item[\textbf{11-4 b.}]



\end{enumerate}



\end{solution}




\end{document}